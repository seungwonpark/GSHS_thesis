\clearpage
% using BiBTeX
%\addcontentsline{toc}{section}{References}
%% onehalfspace 가 안됨...



%-----------------------------------------------------
%   Summary (영어로 작성한 학생은 이 부분 전체를 제거한다.)
%-----------------------------------------------------
%\begin{summary}
%\addcontentsline{toc}{section}{Summary}  %%% TOC에 표시
%한글로 졸업논문을 작성한 학생은 반드시 5페이지 내외의 영어 요약문을 작성해야 합니다. 영문으로 작성하는 학생은 이 부분을 작성하지 않아도 됩니다.
%\end{summary}

%-----------------------------------------------------
%   감사의 글
%-----------------------------------------------------
\begin{acknowledgements}
\addcontentsline{toc}{section}{감사의 글}  %%% TOC에 표시
드디어 제가 난생 처음 그럴듯한 논문을 완성하게 되었습니다. 저 스스로도 뿌듯하지만 주변에서 저를 도와주었던 분들의 도움이 있었기에 가능했던 일이라고 생각됩니다.

먼저, 2015년 한 해동안 저희의 연구 방향을 제시해 주시고 지도하여 주셨으며 이 연구 분야의 큰 field를 볼 수 있도록 도와주신 성균관대학교 문형필 교수님께 감사드립니다. 또한, 조교님의 연구만 해도 바쁜 실정인데 저희의 연구에 관심을 가져 주시고 많은 도움을 주신 Luong Anh Tuan 씨에게 감사드립니다. Luong Anh Tuan 씨 외에도 저희의 연구에 관심을 가져 주시고 조언을 아끼지 않으신 성균관대학교 차세대로봇 액추에이터/센서 연구센터 연구원 분들께 감사드립니다.

저희의 심화 R\&E를 교내에서 지도해 주셨으며 교수님을 소개해 주시고, SRC 500호와 같이 교내에 R\&E를 진행할 고정적인 장소를 마련해 주셨으며 야간에도 연구 및 안전 지도를 해 주시고, 휴먼테크 논문대회 때 도움을 주셨으며, 졸업논문 지도교사까지도 맡아 주신 오정현 선생님께 너무나도 감사드립니다. 또한, 매번 야간에 실험실 사용을 할 수 있도록 허락해 주시고 고가의 실험 물품들을 빌려주시는 등 적극 협조해 주신 우리학교 물리 테크니션 이광원 선생님께도 감사드립니다.

또한, 구두발표 때 피드백을 해 주셨으며, \LaTeX 을 사용하는 학생들을 위해 휴먼테크논문대회와 졸업논문의 \LaTeX 양식을 만들어 주신 목진욱 선생님께도 감사드립니다. % \TeX 관련하여서는, 2014년 당시 저희에게 \TeX 을 가르쳐 주신 세종과학예술영재학교의 정민석 선생님께도 감사드리며, 학습용으로 본인의 졸업논문을 작성하는 데에 사용한 tex 파일을 준 31기 윤지용 선배께도 감사드립니다.

저에게 심화 R\&E를 같이 하자고 제안해 주고, 2년간 심화 R\&E와 졸업논문을 쓰는 과정에서 동료로써 많은 도움을 준 김형주 학생에게 감사합니다. 
%또한, 2014학년도 겨울방학 때 KYPT 실험이 진행되는 실험실을 방문하였을 때, 2015 IYPT Prob No.3 ``Artificial muscle'' 를 소개해 준 Phronesis 팀의 황동욱 학생, 양찬솔 학생에게 감사드립니다. 
`친구를 잘 두어야 한다'라는 말은 이럴 때 쓰라고 있는 말인 것 같습니다.

끝으로, 항상 저에게 항상 큰 은혜를 베풀어 주시는 부모님께 감사드립니다. 경기과학고를 졸업하고 나서도, 사람들에게 한 줄기 빛이 되는 사람이 되도록 노력하겠습니다.
\end{acknowledgements}

%-----------------------------------------------------
%   연구활동 
%-----------------------------------------------------
\begin{researches}
\addcontentsline{toc}{section}{연구활동}  %%% TOC에 표시
\begin{itemize}
\item{2015 한국영재학회 추계학술대회 영재학교 R\&E 및 과학영재교육원 산출물 발표 참가}
\\ - 제목 : SCP Artificial Muscle로 작동하는 Antagonistic Robot Arm의 Feedback 제어
\item{2015학년도 제5회 과학영재학교 우수 R\&E 공동발표회 참가, KAIST 총장상 수상}
\\ - 제목 : SCP Artificial Muscle로 작동하는 Antagonistic Robot Arm의 Feedback 제어
\item{제 22회 휴먼테크논문대회 은상 수상(고교 물리 부문)}
\\ - 제목 : SCP Artificial Muscle로 작동하는 Antagonistic Robot Arm의 Feedback 제어
\item{2015학년도 경기과학고등학교 연구대상 수상}
\\ - 제목 : SCP Artificial Muscle로 작동하는 Antagonistic Robot Arm의 Feedback 제어
\item{제 62회	경기도과학전람회 참가 예정}
\end{itemize}
\end{researches}