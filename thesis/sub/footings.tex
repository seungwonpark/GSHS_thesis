\clearpage
% using BiBTeX
\begin{onehalfspace} %% onehalfspace 가 안됨...
	\bibliographystyle{ieeetr}
	\bibliography{bibfile}
\end{onehalfspace}


%-----------------------------------------------------
%   Summary (영어로 작성한 학생은 이 부분 전체를 제거한다.)
%-----------------------------------------------------
%\begin{summary}
%\addcontentsline{toc}{section}{Summary}  %%% TOC에 표시
%한글로 졸업논문을 작성한 학생은 반드시 5페이지 내외의 영어 요약문을 작성해야 합니다. 영문으로 작성하는 학생은 이 부분을 작성하지 않아도 됩니다.
%\end{summary}

%-----------------------------------------------------
%   감사의 글
%-----------------------------------------------------
\begin{acknowledgements}
\addcontentsline{toc}{section}{감사의 글}  %%% TOC에 표시
먼저, 2015년 한 해동안 저희의 연구 방향을 제시해 주시고 지도하여 주신 성균관대학교 문형필 교수님께 감사드립니다. 또한, 조교님의 연구만 해도 바쁜 실정인데 저희의 연구에 관심을 가져 주시고 많은 도움을 주신 Luong Anh Tuan 씨에게 감사드립니다. 또한, Luong Anh Tuan 씨 외에도 저희의 연구에 관심을 가져 주시고 조언을 아끼지 않으신 성균관대학교 차세대로봇 액추에이터/센서 연구센터 연구원 분들께 감사드립니다.

저희의 심화 R\&E를 교내에서 지도해 주셨으며 교수님을 소개해 주시고, SRC 500호와 같이 교내에 R\&E를 진행할 고정적인 장소를 마련해 주셨으며 야간에도 연구 및 안전 지도를 해 주시고, 휴먼테크 논문대회 때 도움을 주셨으며, 졸업논문 지도교사까지도 맡아 주신 오정현 선생님께 너무나도  감사드립니다.

마지막으로, 먼저 저에게 심화 R\&E를 같이 하자고 제안해 주고, 2년간 심화 R\&E와 졸업논문을 쓰는 과정에서 동료로써 많은 도움을 준 김형주 학생에게 감사합니다. 형주 덕분에 2년간 정말 많은 것을 배울 수 있었던 것 같습니다. `친구를 잘 두어야 한다'라는 말은 이럴 때 쓰라고 있는 말인 것 같습니다.
\end{acknowledgements}

%-----------------------------------------------------
%   연구활동 
%-----------------------------------------------------
\begin{researches}
\addcontentsline{toc}{section}{연구활동}  %%% TOC에 표시
\begin{itemize}
\item{2015 한국영재학회 추계학술대회 영재학교 R\&E 및 과학영재교육원 산출물 발표 참가}
\\ - 제목 : SCP Artificial Muscle로 작동하는 Antagonistic Robot Arm의 Feedback 제어
\item{2015학년도 제5회 과학영재학교 우수 R\&E 공동발표회 참가, KAIST 총장상 수상}
\\ - 제목 : SCP Artificial Muscle로 작동하는 Antagonistic Robot Arm의 Feedback 제어
\item{제 22회 휴먼테크논문대회 은상 수상(고교 물리 부문)}
\\ - 제목 : SCP Artificial Muscle로 작동하는 Antagonistic Robot Arm의 Feedback 제어
\item{2015학년도 경기과학고등학교 연구대상 수상}
\\ - 제목 : SCP Artificial Muscle로 작동하는 Antagonistic Robot Arm의 Feedback 제어
\item{제 62회	경기도과학전람회 참가 예정}
\end{itemize}
\end{researches}

\end{document} 