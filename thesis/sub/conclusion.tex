\section{Conclusion}
Various investigation for method of \scp's efficient control had been done by prior works \cite{haines,mirvakili,yip}. %1
However, these studies hadn't established efficient method for cooling \scp. %2
This could be solved by using antagonism, but yet not sustainable.
In this study, we tested the performance of cooling method with compressed air can, demonstrated the sustainable \apcnospace, and proved it by simulation with established thermodynamical models. %3

We found that thermal conductivity of \scp can be feedback controlled by repeating opening and closing solenoid valve with constant period and variable ratio. %4
Also, this method always showed higher thermal conductivity than cooling by fans.
Therefore, this work extends an achievement of Yip \etal, proving that \apc can be highly sustainable.

Most notably, this is the first study to make thermal conductivity to be feedback controlled. This could be done by extraordinary idea - opening and closing solenoid valve repeatedly.

However, some limitations are worth noting. Required temperature for feedback cooling was too high, which is not energy-efficient.
Also, compressed air cans had to be replaced frequently because pressure of them wasn't constant enough. 
Future work should therefore try the forced air cooling through tube with big capacity such as Helium tank. 
This will also provide higher thermal conductivity because Helium have smaller molecular mass than atmosphere.


%% Tank 보다는 Can 을 사용함으로써 휴대 가능하게 함
%% 하지만 Can 이라서 빨리 소진되는 경향이 있기는 함