\section{Conclusion}
Various investigations for the method of \scpnospace's efficient control have been conducted by preceding researches \cite{haines,mirvakili,yip}. %1
However, these researches haven't established efficient methods for cooling \scpnospace. %2
%This could be solved by using antagonism, but yet not sustainable.
In this study, \antas were made to perform fast response of displacement on two sides, but they were not yet sustainable. Thus, we proved the possibility of sustainable \apc by forced air-cooling through simulations of thermodynamical models. %3

We found that the thermal conductivity of \scp can be feedback controlled by repeating opening and closing of the solenoid valve in variable ratios for a certain period of time. %4
This is utilized by using compressed air, which is known to be a natural method in operating temperature, along with water cooling \cite{madden}.
%What makes this method special is its thermal conductivity that is only exceeded by water cooling.
Also, this method showed higher thermal conductivity than that of cooling by computer fans \cite{yip}, or natural cooling.
Therefore, this study extends the achievement of previous studies.
%, proving that \apc can be highly sustainable.
% ``What attributes sustainability?''

Most notably, this is the first study to make thermal conductivity feedback controlled. 
This could be done by an extraordinary idea - opening and closing of the solenoid valve repeatedly.
Therefore, two of the muscles in \anta were able to be cooled without effecting their temperature difference.
These striking ideas had proved that strategy for sustainable \apc is possible.
%Since hysteresis of \scp is known to be lower at high temperature \cite{moretti}, 
However, some limitations on this study are worth noting. 
%Required temperature for feedback cooling was too high, which is not energy-efficient.
Compressed air cans had to be replaced frequently because the pressure of them wasn't constant enough. 
Also, solenoid valve's operating sound and vibration were too big to be applied into real robots.
Furthermore, since the muscle was electrical conductive, water cooling - which provides higher thermal conductivity than air \cite{finger} - couldn't be utilized.
%Future work should therefore try the forced air cooling through tube with big capacity such as Helium tank. 
Further research should therefore establish efficient, mobile cooling methods and or devices for \scps with larger capacity. 
In addition, extra equipments such as portable air compressors can be used as a cooling device.
%This will also provide higher thermal conductivity since Helium have smaller molecular mass than atmosphere.


%% Tank 보다는 Can 을 사용함으로써 휴대 가능하게 함
%% 하지만 Can 이라서 빨리 소진되는 경향이 있기는 함

% 'Electrical heating combined with compressed air or water for cooling are natural choices in the operating room. : Madden2015 : `Twisted Lines