\section{Conclusion}
Various investigation for method of \scpnospace's efficient control had been done by prior works \cite{haines,mirvakili,yip}. %1
However, these studies hadn't established efficient method for cooling \scpnospace. %2
%This could be solved by using antagonism, but yet not sustainable.
In this study, \anta was made to perform fast response of displacement on two side, but yet not sustainable. Thus, we tested the performance of cooling method with compressed air can, demonstrated the sustainable \apcnospace, and proved it by simulation with thermodynamical models. %3

We found that thermal conductivity of \scp can be feedback controlled by repeating opening and closing solenoid valve with constant period and variable ratio. %4
This is done by using compressed air, which is known to be natural method in operating temperature, along with water cooling \cite{madden}.
Also, this method showed higher thermal conductivity than cooling by computer fans \cite{yip}.
Therefore, this work extends an achievement of Yip \etal, proving that \apc can be highly sustainable.

Most notably, this is the first study to make thermal conductivity to be feedback controlled. This could be done by extraordinary idea - opening and closing solenoid valve repeatedly.
%Since hysteresis of \scp is known to be lower at high temperature \cite{moretti}, 
However, some limitation is worth noting. 
%Required temperature for feedback cooling was too high, which is not energy-efficient.
Compressed air cans had to be replaced frequently because pressure of them wasn't constant enough. 
Also, solenoid valve's operating sound and vibration was too big to be apply into real robots.
%Future work should therefore try the forced air cooling through tube with big capacity such as Helium tank. 
Future work should therefore establish efficient, mobile cooling method/device for \scp with larger capacity. 
Extra equipments such as portable air compressor can be used.
%This will also provide higher thermal conductivity since Helium have smaller molecular mass than atmosphere.


%% Tank 보다는 Can 을 사용함으로써 휴대 가능하게 함
%% 하지만 Can 이라서 빨리 소진되는 경향이 있기는 함

% 'Electrical heating combined with compressed air or water for cooling are natural choices in the operating room. : Madden2015 : `Twisted Lines