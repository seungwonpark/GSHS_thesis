\documentclass[twoside,11pt]{gshs_thesis}
\usepackage{subcaption}
\usepackage[per-mode=symbol]{siunitx}
\sisetup{inter-unit-product =$\cdot$} % http://tex.stackexchange.com/questions/59032/how-to-format-si-units
%\usepackage[hidelinks]{hyperref}
\usepackage{verbatim}
\usepackage{indentfirst}
\usepackage{diagbox}
\captionsetup{subrefformat=parens}
\graphicspath{{../images/}}
%\captionsetup[figure]{font=small,skip=10pt}
%\renewcommand\figurename{Fig.}
\usepackage{enumitem}
\setitemize{noitemsep} % for reducing line space in itemize environments
\setenumerate{noitemsep} % for reducing line space in enumerate environments
%\usepackage[hidelinks]{hyperref}
%\hypersetup{pdfencoding=auto,bookmarks}
%\usepackage{bookmark}
%\usepackage{tocloft}
%\cftpagenumbersoff{section}
%\setcounter{secnumdepth}{2} % toc에는 subsubsection을 제외
%\setcounter{tocdepth}{2}
\usepackage{array}
%\usepackage{multirow}
\newcommand{\nocontentsline}[3]{}
\newcommand{\tocless}[2]{\bgroup\let\addcontentsline=\nocontentsline#1{#2}\egroup}

\newcommand{\etal}{{\it et al.}}
\newcommand{\etalspace}{{\it et al. }}
\newcommand{\scp}{SCP artificial muscle }
\newcommand{\SCP}{SCP Artificial Muscle }
\newcommand{\ANTA}{Antagonistic Robot Arm }
\newcommand{\Anta}{Antagonistic robot arm } 
\newcommand{\anta}{antagonistic robot arm }
\newcommand{\APC}{Antagonistic Position Control }
\newcommand{\Apc}{Antagonistic position control }
\newcommand{\apc}{antagonistic position control }
\newcommand{\scpnospace}{SCP artificial muscle} 
\newcommand{\SCPnospace}{SCP Artificial Muscle}
\newcommand{\ANTAnospace}{Antagonistic Robot Arm}
\newcommand{\Antanospace}{Antagonistic robot arm} 
\newcommand{\antanospace}{antagonistic robot arm}
\newcommand{\APCnospace}{Antagonistic Position Control}
\newcommand{\Apcnospace}{Antagonistic position control}
\newcommand{\apcnospace}{antagonistic position control}
\newcommand{\needcitation}{[Citation Needed]}
\newcommand{\scps}{SCP artificial muscles }
\newcommand{\scpsnospace}{SCP artificial muscles}
\newcommand{\ANTAs}{Antagonistic Robot Arms } 
\newcommand{\antas}{antagonistic robot arms }
\newcommand{\antasnospace}{antagonistic robot arms}
\newcommand{\Antas}{Antagonistic robot arms }
\newcommand{\Antasnospace}{Antagonistic robot arms}

% -----------------------------------------------------------------------
%                   이 부분은 수정하지 마시오.
% -----------------------------------------------------------------------
\titleheader{졸업논문청구논문}
\school{과학영재학교 경기과학고등학교}
\approval{위 논문은 과학영재학교 경기과학고등학교 졸업논문으로\\
졸업논문심사위원회에서 심사 통과하였음.}
\chairperson{심사위원장}
\examiner{심사위원}
\apprvsign{(인)}
\korabstract{초 록}
\koracknowledgement{감사의 글}
\korresearches{연 구 활 동}

%: ----------------------------------------------------------------------
%:                  논문 제목과 저자 이름을 입력하시오
% ----------------------------------------------------------------------
\title{SCP로 작동하는 Antagonistic Robot Arm의 공냉에 의한 지속 가능한 제어} %한글 제목
\engtitle{SCP-Powered Antagonistic Robot Arm's \\ Sustainable Control by Forced Air-Cooling} %영문 제목
\korname{박 승 원} %저자 이름을 한글로 입력하시오 (글자 사이 띄어쓰기)
\engname{Park, Seung-Won} %저자 이름을 영어로 입력하시오 (family name, personal name)
\chnname{朴 承 元} %저자 이름을 한자로 입력하시오 (글자 사이 띄어쓰기)
\studid{14041} %학번을 입력하시오

%------------------------------------------------------------------------
%                  심사위원과 논문 승인 날짜를 입력하시오
%------------------------------------------------------------------------
\advisor{Oh, Jeounghyun}  %지도교사 영문 이름 (family name, personal name)
\judgeone{교 수 님} %심사위원장
\judgetwo{남 태 정}   %심사위원1
\judgethree{오 정 현} %심사위원2(지도교사)
\degreeyear{2017}   %졸업 년도
\degreedate{2016}{X}{X} %논문 승인 날짜 양식1
%\degreedatekor{2016년 X월 X일} %논문 승인 날짜 양식2

%------------------------------------------------------------------------
%                  논문제출 전 체크리스트를 확인하시오
%------------------------------------------------------------------------
\checklisttitle{[논문제출 전 체크리스트]} %수정하지 마시오
\checklistI{1. 이 논문은 내가 직접 연구하고 작성한 것이다.} %수정하지 마시오
% 이 항목이 사실이라면 다음 줄 앞에 "%"기호 삽입, 다다음 줄 앞의 "%"기호 제거하시오
%\checklistmarkI{$\square$}
\checklistmarkI{$\text{\rlap{$\checkmark$}}\square$}
\checklistII{2. 인용한 모든 자료(책, 논문, 인터넷자료 등)의 인용표시를 바르게 하였다.} %수정하지 마시오
% 이 항목이 사실이라면 다음 줄 앞에 "%"기호 삽입, 다다음 줄 앞의 "%"기호 제거하시오
%\checklistmarkII{$\square$}
\checklistmarkII{$\text{\rlap{$\checkmark$}}\square$}
\checklistIII{3. 인용한 자료의 표현이나 내용을 왜곡하지 않았다.} %수정하지마시오
% 이 항목이 사실이라면 다음 줄 앞에 "%"기호 삽입, 다다음 줄 앞의 "%"기호 제거하시오
%\checklistmarkIII{$\square$}
\checklistmarkIII{$\text{\rlap{$\checkmark$}}\square$}
\checklistIV{4. 정확한 출처제시 없이 다른 사람의 글이나 아이디어를 가져오지 않았다.} %수정하지 마시오
% 이 항목이 사실이라면 다음 줄 앞에 "%"기호 삽입, 다다음 줄 앞의 "%"기호 제거하시오
%\checklistmarkIV{$\square$}
\checklistmarkIV{$\text{\rlap{$\checkmark$}}\square$}
\checklistV{5. 논문 작성 중 도표나 데이터를 조작(위조 혹은 변조)하지 않았다.} %수정하지 마시오
% 이 항목이 사실이라면 다음 줄 앞에 "%"기호 삽입, 다다음 줄 앞의 "%"기호 제거하시오
%\checklistmarkV{$\square$}
\checklistmarkV{$\text{\rlap{$\checkmark$}}\square$}
\checklistVI{6. 다른 친구와 같은 내용의 논문을 제출하지 않았다.} %수정하지 마시오
% 이 항목이 사실이라면 다음 줄 앞에 "%"기호 삽입, 다다음 줄 앞의 "%"기호 제거하시오
%\checklistmarkVI{$\square$}
\checklistmarkVI{$\text{\rlap{$\checkmark$}}\square$}

