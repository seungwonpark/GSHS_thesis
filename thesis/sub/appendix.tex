
\clearpage  %%% Appendix를 새 페이지에서 시작
\appendix
\renewcommand{\thesection}{\Alph{section}} %%% TOC에 appendix numbering 재설정
\renewcommand{\thesubsection}{\arabic{subsection}}
\renewcommand{\thesubsubsection}{\arabic{subsubsection}}
\titleformat{\section}[hang] {\normalfont\LARGE\bfseries}{\Alph{section}.}{1em}{} %%% Appendix section title의 재설정
\titleformat{\subsection}[hang] {\normalfont\Large\bfseries}{\Alph{section}.\arabic{subsection}.}{1em}{}
\titleformat{\subsubsection}[hang] {\normalfont\bfseries}{\Alph{section}.\arabic{subsection}.\arabic{subsubsection}.}{1em}{}
\renewcommand{\theequation}{\thesection.\arabic{equation}} %%% Appendix equation numbering 의 재설정
\renewcommand{\thefigure}{\thesection-\arabic{figure}} %%% Appendix figure numbering 의 재설정
\renewcommand{\thetable}{\thesection-\arabic{table}} %%% Appendix table numbering 의 재설정
\setcounter{equation}{0} %%% Appendix equation starting number의 초기화
\setcounter{figure}{0} %%% Appendix figure starting number의 초기화
\setcounter{table}{0} %%% Appendix table starting number의 초기화
\section{List of Used Materials}
\begin{table}[h]
	\caption{List of used materials}
	\label{used_materials}
	\begin{center}
		\begin{tabular}{m{0.28\textwidth}||m{0.2\textwidth}|m{0.2\textwidth}|m{0.22\textwidth}}
			\hline
			Item & Name & Manufacturer & Note \\
			\hline
			\hline
			\multicolumn{4}{l}{Sensors} \\ \hline
			Temperature sensor & \small{TC1047AVNB} & Microchip & \small{\SI{-40}{\degreeCelsius}-\SI{125}{\degreeCelsius}} \\
			\hline
			Rotary sensor & \small{SV01A103 AEA01B00} & Murata & Frictional torque : \SI{2}{\milli\newton \meter}\\
			\hline
			Slide potentiometer & PTA2043 & Bourns & Maximum displacement : \SI{20}{\milli\meter}\\
			\hline
			Load cell & CB1A & Dacell & CAPA : \SI{3}{\kg f} \\
			\hline
			Amplifier & DN-AM100 & Dacell & Amplification X100 - X1500 \\
			\hline
			Arduino Uno & Arduino Uno & arduino.cc & Certified \\
			\hline
			Data acquisition system & \small{NI cRIO-9024} & \small{National Instruments} & Maximum voltage : \SI{10}{\volt} \\
			\hline
			\hline
			\multicolumn{4}{l}{Actuating system} \\ \hline
			Conductive nylon thread & \small{Conductive Yarn} & Shieldex & Size 92\\
			\hline
			{\bf Compressed air tank} & Dr.99 & \small{BEX Intercorportation} & Nonflammable \\
			\hline
			{\bf Solenoid valve} & HSV-FF & StormTec & High-pressure, DC 12V \\
			\hline
			{\bf Computer fan} & \small{DF0601512SEU2F} & Cool Flow & \SI{12}{\volt}, Size : 60 X 60 X 15 \si{\milli\meter}\\
			\hline
			\hline
			\multicolumn{4}{l}{\Anta} \\ \hline
			MOSFET & IRFZ44N & International Rectifier & Maximum current : \SI{35}{\ampere}\\
			\hline
			PCB & PCB & Devicemart & custom service \\
			\hline
			Ball bearing & 6000 & KBC & \small{Outside/Inside diameter : \SI{26}{\milli\meter}, \SI{10}{\milli\meter}}\\
			\hline
		\end{tabular}
	\end{center}
\end{table}

\section{Arduino Codes}\label{code_dynamic}
Due to long length, only code for dynamic experiment in section \ref{section_dynamic} is shown. Other codes for \apc and simulation in section \ref{section_simulation} is accessible in GitHub.
\footnote{URL : https://github.com/seungwonpark/thesis\_arduino}

\begin{scriptsize}
\begin{verbatim}
// Measuring Thermal conductivity of air flow induced by Compressed air tank and Arduino-controlled Single Solenoid Valve
// In this experiment, supplied power on muscle is constant
#include "math.h"
const int solvPin1 = 3; // Solenoid valve pin (MOSFET)
const int tempPin1 = A1; // Temperature sensor pin
const double pi = 3.1416;
int tRead1;
bool start = false;
int starttime;
unsigned long time;
double t1(){return 125.0*tRead1/256.0-50.0;} // in degrees Celsius

const double r1 = 2.5; // resistance of actuator : 2.5 ohm
const double V1 = 5.00; // Voltage of Power Supply : Constant
double ratio = 0;// Time ratio(0 - 1) of SV open : input from serial
double interval = 100.0; // According to experiments, near 50ms is the shortest response time.

void setup() {
Serial.begin(9600);
pinMode(solvPin1,OUTPUT);
}

void loop() {
int serial_input;
if(Serial.available()){ 
serial_input = Serial.read(); // By typing 0 - 9 + "Enter", start SV
serial_input -= 48; // `0' - `9' and `:'(colon) : 48 - 57 and 58. Refer to Ascii table.
ratio = serial_input * 0.1;
start = true;
starttime = millis();
}
if(start == true){
time = millis() - starttime;
tRead1 = analogRead(tempPin1);
dataPrint();
SVsignal();
}
}

void dataPrint(){

Serial.print(time);
Serial.print(" ");
Serial.print(t1());
Serial.print(" ");
Serial.println(" ");
}
void SVsignal(){
digitalWrite(solvPin1, HIGH); // SV doesn't have to be analog controlled
delay(interval*ratio);
digitalWrite(solvPin1, LOW);
delay(interval*(1-ratio));
}
\end{verbatim}
\end{scriptsize}
